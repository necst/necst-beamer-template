\documentclass[xcolor=dvipsnames,compress]{beamer}
\usepackage{CJK}
\usepackage[italian]{babel}
\usepackage[latin1]{inputenc}
\usepackage[T1]{fontenc}

%\definecolor{necstgreen}{rgb}{0.137,0.466,0.741}
%\definecolor{necstgreen}{RGB}{240,40,40}

%\definecolor{necstgreen}{RGB}{187,216,113}
\definecolor{necstgreen}{RGB}{171,216,80}
\definecolor{necstgreen2}{RGB}{0,158, 96}
\definecolor{necstgray}{RGB}{30,30,30}

\usetheme{Madrid}
\useoutertheme[subsection=false]{miniframes}
\useinnertheme{rectangles}
\usecolortheme[named=necstgreen2]{structure}
\setbeamercolor*{palette primary}{use=structure,fg=black,bg=necstgreen}
%\setbeamertemplate{items}[ball,bg=necstgreen] 
%\setbeamertemplate{blocks}[rectangles][shadow=false] 
\setbeamertemplate{items}[circle]

\defbeamertemplate*{title page}{customized}[1][]
{
  \vbox{}
  \vfill
  \begin{centering}
    \begin{beamercolorbox}[sep=8pt,center,rounded=true,shadow=true]{title}
      \usebeamerfont{title}\inserttitle\par%
      \ifx\insertsubtitle\@empty%
      \else%
        \vskip0.25em%
        {\usebeamerfont{subtitle}\usebeamercolor[fg]{subtitle}\insertsubtitle\par}%
      \fi%     
    \end{beamercolorbox}%
    \vskip1em
    \vfill
    \par
    \begin{beamercolorbox}[sep=8pt,center,#1]{author}
      \usebeamerfont{author}\textbf{Tesi di Laurea Magistrale di} \\ \insertauthor
    \end{beamercolorbox}
    \begin{beamercolorbox}[sep=8pt,center,#1]{institute}
      \usebeamerfont{institute}\insertinstitute
    \end{beamercolorbox}
    \begin{beamercolorbox}[sep=8pt,center,#1]{date}
      \usebeamerfont{date}\insertdate
    \end{beamercolorbox}\vskip0.5em
    {\usebeamercolor[fg]{titlegraphic}\inserttitlegraphic\par}
  \end{centering}
}


\AtBeginSection[]{
\begin{frame}
	\begin{center}
		\begin{beamercolorbox}[sep=8pt,center,rounded=true,shadow=true]{part title}
        \usebeamerfont{part title}\insertsection
     \end{beamercolorbox}
\end{center}
\end{frame}
}

\setbeamercolor{section in head/foot}{fg=white, bg=necstgray}
\setbeamercolor{author in head/foot}{fg=black, bg=necstgreen}
\setbeamercolor{logo in head/foot}{fg=white, bg=necstgray}
\setbeamercolor{page number in head/foot}{fg=black, bg=necstgreen}

%Color itemize/description ...
\setbeamercolor{description item}{fg=necstgreen2}

\setbeamertemplate{footline}
{
  \leavevmode%
  \hbox{%
  \begin{beamercolorbox}[wd=.10\paperwidth,ht=2.25ex,dp=1ex,center]{logo in head/foot}%
    \usebeamerfont{logo in head/foot}NECSTLab
  \end{beamercolorbox}%
  \begin{beamercolorbox}[wd=.20\paperwidth,ht=2.25ex,dp=1ex,center]{author in head/foot}%
    \usebeamerfont{author in head/foot}\insertshortauthor
  \end{beamercolorbox}%
  \begin{beamercolorbox}[wd=.65\paperwidth,ht=2.25ex,dp=1ex,center]{title in head/foot}%
    \usebeamerfont{title in head/foot}\insertshorttitle
  \end{beamercolorbox}}%
  \begin{beamercolorbox}[wd=.05\paperwidth,ht=2.25ex,dp=1ex,center]{page number in head/foot}%
    \usebeamerfont{page number in head/foot}\insertframenumber
  \end{beamercolorbox}%
  \vskip0pt%
}


\begin{document}
\setbeamertemplate{navigation symbols}{}

\title[ZvH]{Zorro vs Hackers:\\ Storia di un Vero Amore}
		
\author[P. Pluto, T. Caio]{Pippo Pluto, \small{matr. xxxxxx}\\ Tizio Caio, \small{matr. yyyyyy}\\~\\ \textbf{Relatore:} Prof. Pico de la Mirandola \\ \textbf{Correlatore:}El Se\~nor Diego de la Vega\vspace{0.3cm}}

\institute{
\begin{columns} 
    \begin{column}{0.17\columnwidth}
        \includegraphics[width=\columnwidth]{img/PoliMI.eps}\hfill
    \end{column}
    \begin{column}{0.65\columnwidth}
        \centering \scriptsize{Politecnico di Milano\\Scuola di Ingegneria Industriale e dell'Informazione\\Corso di Laurea Specialistica in Ingegneria Informatica}
    \end{column}
    \begin{column}{0.17\columnwidth}
        \hfill\includegraphics[scale=0.8]{img/necst-official-logo-white.pdf}
    \end{column}
\end{columns}
}

\date{}

\frame[plain]{\vspace{1.1cm}\titlepage} 

\section[Section 1]{Section 1}
\subsection{}
\frame{\frametitle{Title 1} \pause
Trova le \textbf{Differenze}.

\begin{figure}
\centering
\includegraphics[scale=0.5]{img/zorrovsdiego.png}
\end{figure}
\pause \vspace{-0.2cm}
Description:
\begin{description}
\item[Desc1]
\item[Desc2]
\end{description}
}

\section[Section 2]{Section}
\subsection{}

\frame{\frametitle{Title 2} \pause
\begin{columns}
\begin{column}{0.6\textwidth}
\visible<2->{Show 1} \\
\vspace{0.5cm}
\visible<3->{Show 2} \\
\vspace{0.5cm}
\visible<4->{Show 3}
\end{column}
\begin{column}{0.4\textwidth}
\centering
\includegraphics[scale=0.55]{img/zorrovsdiego.png}
\end{column}
\end{columns}
}

\frame{\frametitle{Filtro linguistico: costruzione} \pause

\begin{block}{Block}
Bla bla bla
\end{block}
}

\frame{\frametitle{Conclusioni} \pause
In conclusione, questa presentazione \`e fantastica: \pause
\begin{enumerate}
\item Item 1 \pause
\item Item 2 \pause
\item Utem 3
\end{enumerate}
}


\section{}
\begin{frame}
Grazie per l'attenzione. \textbf{Domande?} \\
\vspace{2cm}
Questo lavoro \`e stato inviato per la pubblicazione a:\\~\\
\textbf{Conferenza}\\
Dicembre 00, Lapponia (Chiedere di Babbo).
\end{frame}

\end{document}
